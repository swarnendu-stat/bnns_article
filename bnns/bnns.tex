\documentclass[
]{jss}

%% recommended packages
\usepackage{orcidlink,thumbpdf,lmodern}

\usepackage[utf8]{inputenc}

\author{
Swarnendu Chatterjee~\orcidlink{0000-0000-0000-0000}\\GSK
}
\title{Bayesian Neural Networks in R: The \pkg{bnns} Package}

\Plainauthor{Swarnendu Chatterjee}
\Plaintitle{Bayesian Neural Networks in R: The bnns Package}
\Shorttitle{\pkg{bnns}: Bayesian Neural Networks in R}


\Abstract{
Bayesian Neural Networks (BNNs) combine the flexibility of neural
networks with the principled uncertainty quantification of Bayesian
methods, making them powerful tools for predictive modeling and
decision-making under uncertainty. The bnns package provides an R
interface for building and training BNNs using the probabilistic
programming capabilities of Stan. With a formula-based interface,
customizable priors, and support for various activation and output
functions, bnns enables users to model complex relationships in
regression and classification tasks. Additionally, the package offers
posterior summaries for parameters and predictions, aiding
interpretability and probabilistic reasoning. Designed for small to
moderately sized datasets, bnns is particularly well-suited for
applications in clinical research, finance, and other domains requiring
robust uncertainty quantification. This article presents the
implementation of the bnns package, its core features, and benchmarking
results, highlighting its utility and performance across diverse machine
learning tasks.
}

\Keywords{bayesian neural networks, probabilistic modeling, uncertainty
quantification, machine learning, \proglang{R}}
\Plainkeywords{bayesian neural networks, probabilistic
modeling, uncertainty quantification, machine learning, R}

%% publication information
%% \Volume{50}
%% \Issue{9}
%% \Month{June}
%% \Year{2012}
%% \Submitdate{}
%% \Acceptdate{2012-06-04}

\Address{
    Swarnendu Chatterjee\\
    GSK\\
    First line\\
Second line\\
  E-mail: \email{swarnendu.stat@gmail.com}\\
  URL: \url{https://posit.co}\\~\\
  }


% tightlist command for lists without linebreak
\providecommand{\tightlist}{%
  \setlength{\itemsep}{0pt}\setlength{\parskip}{0pt}}




\usepackage{amsmath}

\begin{document}



\section{Introduction}\label{introduction}

This template demonstrates some of the basic LaTeX that you need to know
to create a JSS article.

\subsection{Code formatting}\label{code-formatting}

In general, don't use Markdown, but use the more precise LaTeX commands
instead:

\begin{itemize}
\item
  \proglang{Java}
\item
  \pkg{plyr}
\end{itemize}

One exception is inline code, which can be written inside a pair of
backticks (i.e., using the Markdown syntax).

If you want to use LaTeX commands in headers, you need to provide a
\texttt{short-title} attribute. You can also provide a custom identifier
if necessary. See the header of Section \ref{r-code} for example.

\section[R code]{\proglang{R} code}\label{r-code}

Can be inserted in regular R markdown blocks.

\begin{CodeChunk}
\begin{CodeInput}
R> x <- 1:10
R> x
\end{CodeInput}
\begin{CodeOutput}
 [1]  1  2  3  4  5  6  7  8  9 10
\end{CodeOutput}
\end{CodeChunk}

\subsection[Features specific to rticles]{Features specific to
\pkg{rticles}}\label{features-specific-to}

\begin{itemize}
\tightlist
\item
  Adding short titles to section headers is a feature specific to
  \pkg{rticles} (implemented via a Pandoc Lua filter). This feature is
  currently not supported by Pandoc and we will update this template if
  \href{https://github.com/jgm/pandoc/issues/4409}{it is officially
  supported in the future}.
\item
  Using the \texttt{\textbackslash{}AND} syntax in the \texttt{author}
  field to add authors on a new line. This is a specific to the
  \texttt{rticles::jss\_article} format.
\end{itemize}




\end{document}
